\documentclass[english,floatsintext,man]{apa6}

\usepackage{amssymb,amsmath}
\usepackage{ifxetex,ifluatex}
\usepackage{fixltx2e} % provides \textsubscript
\ifnum 0\ifxetex 1\fi\ifluatex 1\fi=0 % if pdftex
  \usepackage[T1]{fontenc}
  \usepackage[utf8]{inputenc}
\else % if luatex or xelatex
  \ifxetex
    \usepackage{mathspec}
    \usepackage{xltxtra,xunicode}
  \else
    \usepackage{fontspec}
  \fi
  \defaultfontfeatures{Mapping=tex-text,Scale=MatchLowercase}
  \newcommand{\euro}{€}
\fi
% use upquote if available, for straight quotes in verbatim environments
\IfFileExists{upquote.sty}{\usepackage{upquote}}{}
% use microtype if available
\IfFileExists{microtype.sty}{\usepackage{microtype}}{}

% Table formatting
\usepackage{longtable, booktabs}
\usepackage{lscape}
% \usepackage[counterclockwise]{rotating}   % Landscape page setup for large tables
\usepackage{multirow}		% Table styling
\usepackage{tabularx}		% Control Column width
\usepackage[flushleft]{threeparttable}	% Allows for three part tables with a specified notes section
\usepackage{threeparttablex}            % Lets threeparttable work with longtable

% Create new environments so endfloat can handle them
% \newenvironment{ltable}
%   {\begin{landscape}\begin{center}\begin{threeparttable}}
%   {\end{threeparttable}\end{center}\end{landscape}}

\newenvironment{lltable}
  {\begin{landscape}\begin{center}\begin{ThreePartTable}}
  {\end{ThreePartTable}\end{center}\end{landscape}}




% The following enables adjusting longtable caption width to table width
% Solution found at http://golatex.de/longtable-mit-caption-so-breit-wie-die-tabelle-t15767.html
\makeatletter
\newcommand\LastLTentrywidth{1em}
\newlength\longtablewidth
\setlength{\longtablewidth}{1in}
\newcommand\getlongtablewidth{%
 \begingroup
  \ifcsname LT@\roman{LT@tables}\endcsname
  \global\longtablewidth=0pt
  \renewcommand\LT@entry[2]{\global\advance\longtablewidth by ##2\relax\gdef\LastLTentrywidth{##2}}%
  \@nameuse{LT@\roman{LT@tables}}%
  \fi
\endgroup}


\ifxetex
  \usepackage[setpagesize=false, % page size defined by xetex
              unicode=false, % unicode breaks when used with xetex
              xetex]{hyperref}
\else
  \usepackage[unicode=true]{hyperref}
\fi
\hypersetup{breaklinks=true,
            pdfauthor={},
            pdftitle={Evolution of Statistical Software and Quantitative Methods in Educational Research},
            colorlinks=true,
            citecolor=blue,
            urlcolor=blue,
            linkcolor=blue,
            pdfborder={0 0 0}}
\urlstyle{same}  % don't use monospace font for urls

\setlength{\parindent}{0pt}
%\setlength{\parskip}{0pt plus 0pt minus 0pt}

\setlength{\emergencystretch}{3em}  % prevent overfull lines

\ifxetex
  \usepackage{polyglossia}
  \setmainlanguage{}
\else
  \usepackage[english]{babel}
\fi

% Manuscript styling
\captionsetup{font=singlespacing,justification=justified}
\usepackage{csquotes}
\usepackage{upgreek}



\usepackage{tikz} % Variable definition to generate author note

% fix for \tightlist problem in pandoc 1.14
\providecommand{\tightlist}{%
  \setlength{\itemsep}{0pt}\setlength{\parskip}{0pt}}

% Essential manuscript parts
  \title{Evolution of Statistical Software and Quantitative Methods in
Educational Research}

  \shorttitle{Evolution Software and Methods}


  \author{Brandon LeBeau\textsuperscript{1}~\& Ariel Aloe\textsuperscript{1}}

  \def\affdep{{"", ""}}%
  \def\affcity{{"", ""}}%

  \affiliation{
    \vspace{0.5cm}
          \textsuperscript{1} University of Iowa  }

 % If no author_note is defined give only author information if available
      \newcounter{author}
                              \authornote{
            Correspondence concerning this article should be addressed to Brandon LeBeau, Psychological and Quantitative Foundations, University of Iowa, Iowa
City, IA 52245. E-mail: \href{mailto:brandon-lebeau@uiowa.edu}{\nolinkurl{brandon-lebeau@uiowa.edu}}
          }
                                  

  \abstract{Statistical software is the cornerstone of quantitative research studies
and the availability and use of the software can greatly shape which
methods are used by researchers. Software that is more accessible is
likely to have more users and the methods implemented within the
software limits the methods accessible to researchers. Open source
software, (e.g.~R), has reduced these barriers by making cutting edge
statistical methods available to researchers through add-on packages. In
addition, the idea of reproducible analyses has grown significantly
within the statistics and medicine disciplines. This paper aims to
explore the evolution of statistical software within educational
research using a research synthesis to establish the state of affairs.}
  \keywords{Research Synthesis, Statistical Software, Quantitative Methods \\

    
  }





\usepackage{amsthm}
\newtheorem{theorem}{Theorem}
\newtheorem{lemma}{Lemma}
\theoremstyle{definition}
\newtheorem{definition}{Definition}
\newtheorem{corollary}{Corollary}
\newtheorem{proposition}{Proposition}
\theoremstyle{definition}
\newtheorem{example}{Example}
\theoremstyle{remark}
\newtheorem*{remark}{Remark}
\begin{document}

\maketitle

\setcounter{secnumdepth}{0}



\section{Objectives}\label{objectives}

The purpose of this paper is to explore the evolution (or lack thereof)
of statistical software usage over time. As this usage is likely tied
closely to the methods they are employing, the interaction between
software usage quantitative research methods will also be explored.
Research synthesis methods will be used to explore these trends over
time in published educational research journals.

\section{Theoretical Framework}\label{theoretical-framework}

\subsection{Research Questions}\label{research-questions}

\begin{enumerate}
\def\labelenumi{\arabic{enumi}.}
\tightlist
\item
  To what extent has the statistical software usage shifted over time in
  published analyses?

  \begin{itemize}
  \tightlist
  \item
    If there is evidence of a shift, is there evidence this shift
    differs based on quantitative method or journal?
  \end{itemize}
\item
  To what extent are published analyses citing statistical software?

  \begin{itemize}
  \tightlist
  \item
    Has this changed over time and across journals?
  \end{itemize}
\item
  To what extent are open-source software tools used?

  \begin{itemize}
  \tightlist
  \item
    Is there evidence of reproducible analyses being employed?
  \end{itemize}
\end{enumerate}

\section{Methods}\label{methods}

Research synthesis methods will be used to explore the evolution of
statistical software and quantitative methods in educational research.
More specifically, the statistical software used for the analysis will
be coded in additional to the specific quantitative methods (i.e.~linear
regression, hierarchical linear model, etc.). Additional meta data will
also be coded including, journal, article title, author information,
article keywords, and year published. This information will be used to
explore descriptive trends in the data over time, by journals, and
methods.

The research synthesis will gather data from a handful of education
journals that primarily publish empirical data analysis research. The
search will not include journals that the primary focus is
methodological, the use of software in these journals would likely be a
different population. Therefore the following journals were selected to
be searched from 1995 onward:

\begin{itemize}
\tightlist
\item
  American Educational Research Journal
\item
  Educational Researcher
\item
  Educational Evaluation and Policy Analysis
\item
  Higher Education
\item
  Journal of Educational Psychology
\item
  Journal of Experimental Education
\item
  Journal of Teacher Education
\item
  Journal for Research in Mathematics Education
\item
  Sociology of Education
\end{itemize}

\section{Data and Software}\label{data-and-software}

All journal articles published between 1995 through 2016 will be
organized into EndNote. Within EndNote, the find pdf feature will be
used to gather the published documents from each journal. This pdf
database will then be searched using the \emph{pdfsearch} R package
(LeBeau, 2016, R Core Team (2017)). This package allows for keyword
searching directly within pdf documents. This will be the primary data
collection method. The software keywords searched for can be seen in
Table \ref{tab:searchwords}.

\begin{longtable}[]{@{}ll@{}}
\caption{\label{tab:searchwords} Search keywords used in search of published
journal documents}\tabularnewline
\toprule
\begin{minipage}[b]{0.21\columnwidth}\raggedright\strut
Search\strut
\end{minipage} & \begin{minipage}[b]{0.38\columnwidth}\raggedright\strut
Keywords\strut
\end{minipage}\tabularnewline
\midrule
\endfirsthead
\toprule
\begin{minipage}[b]{0.21\columnwidth}\raggedright\strut
Search\strut
\end{minipage} & \begin{minipage}[b]{0.38\columnwidth}\raggedright\strut
Keywords\strut
\end{minipage}\tabularnewline
\midrule
\endhead
\begin{minipage}[t]{0.21\columnwidth}\raggedright\strut
Software\strut
\end{minipage} & \begin{minipage}[t]{0.38\columnwidth}\raggedright\strut
\enquote{SPSS Statistics}, \enquote{SPSS Modeler}, \enquote{SPSS},
\enquote{R}, \enquote{R-project}, \enquote{R project}, \enquote{SAS},
\enquote{JMP}, \enquote{STATA}, \enquote{MATLAB}, \enquote{Statistica},
\enquote{Statsoft}, \enquote{Java}, \enquote{Hadoop}, \enquote{Python} ,
\enquote{Minitab}, \enquote{Systat}, \enquote{Tableau}, \enquote{Scala},
\enquote{Julia}, \enquote{Azure Machine Learning}, \enquote{Mplus},
\enquote{LISREL}, \enquote{AMOS}, \enquote{BILOG}, \enquote{BILOG-MG},
\enquote{Multilog}, \enquote{PARSCALE}, \enquote{IRT PRO},
\enquote{HLM{[}0-9{]}}, \enquote{HLM {[}0-9{]}}\strut
\end{minipage}\tabularnewline
\begin{minipage}[t]{0.21\columnwidth}\raggedright\strut
Quantitative Methods\strut
\end{minipage} & \begin{minipage}[t]{0.38\columnwidth}\raggedright\strut
Still formulating this list\strut
\end{minipage}\tabularnewline
\bottomrule
\end{longtable}

A handful of articles will be randomly selected to be coded manually by
reading the document to evaluate the accuracy of coding from the
\emph{pdfsearch} package.

\section{Initial Results}\label{initial-results}

\section{Scholarly Significance}\label{scholarly-significance}

\section*{References}\label{references}
\addcontentsline{toc}{section}{References}

\hypertarget{refs}{}
\hypertarget{ref-pdfsearch}{}
LeBeau, B. (2016). \emph{Pdfsearch: Search tools for pdf files}.
Retrieved from \url{https://github.com/lebebr01/pdfsearch}

\hypertarget{ref-rpro}{}
R Core Team. (2017). \emph{R: A language and environment for statistical
computing}. Vienna, Austria: R Foundation for Statistical Computing.
Retrieved from \url{https://www.R-project.org/}






\end{document}
